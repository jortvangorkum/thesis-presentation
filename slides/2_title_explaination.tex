\section{Title Explanation}

\begin{slide}{Incremental Computation}
\centering
\large \textbf{Generic \textcolor{red}{Incremental Computation} for Regular Datatypes}

\vspace*{.5cm}
\textbf{=}
\vspace*{.5cm}

Incremental computation is an approach to improve performance by reusing results of a previously computed result when both of the inputs are equal

\end{slide}

\begin{slide}{Generic}
\centering
\large \textbf{\textcolor{red}{Generic} Incremental Computation for Regular Datatypes}

\vspace*{.5cm}
\textbf{=}
\vspace*{.5cm}

Generic refers to \textit{datatype-generic programming}, which is a form of abstraction that allows defining functions that can operate on a large class of datatypes. 

\end{slide}

\begin{slide}{Regular Datatypes}
\centering
\large \textbf{Generic Incremental Computation for \textcolor{red}{Regular Datatypes}}

\vspace*{.5cm}
\textbf{=}
\vspace*{.5cm}

Regular datatypes are recursive datatypes, which can only recurse into themselves, such as lists, binary trees, etc. 

\end{slide}