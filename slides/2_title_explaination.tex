\section{Title Explanation}

% Incremental Computation

\begin{slide}{Incremental Computation}
\centering
\large \textbf{Generic \textcolor{red}{Incremental Computation} for Regular Datatypes}

\vspace*{.5cm}
\textbf{=}
\vspace*{.5cm}

Incremental computation is an approach to improve performance by only recomputing result for changed input

\end{slide}

% Incremental Computation Example

\begin{slide}{Example Incremental Computation}
\begin{haskell}
fib :: Int -> Int
fib 0 = 0
fib 1 = 1
fib n = fib (n - 1) + fib (n - 2)
\end{haskell}
\begin{center}  
\textbf{Call Hierarchy}

\vspace*{.2cm}
\begin{minted}{text}
                  fib(4)  
               /          \
          fib(3)           fib(2)
          /   \            /   \
     fib(2)   fib(1)  fib(1)   fib(0)
     /   \
 fib(1)  fib(0)
\end{minted}
\end{center}
\end{slide}

\begin{slide}{Example Incremental Computation}

\centering
A technique for implementing incremental computations is:

\vspace*{0.4cm}

{\huge \textit{Memoization}}

\vspace*{0.4cm}

stores the result of a computation and returns the cached result when the same input occurs again.

\end{slide}

\begin{slide}{Example Incremental Computation}
\begin{minipage}{.45\textwidth}
\begin{center}
\textbf{New Call Hierarchy}

\vspace*{.7cm}
\begin{minted}[escapeinside=!!]{text}
                  fib(4)  
               /          \
          fib(3)           !\textcolor{red}{fib(2)}!
          /   \      
      fib(2)   fib(1)
      /   \
  fib(1)  fib(0)
\end{minted}
\end{center}
\end{minipage}
\hfill
\begin{minipage}{.45\textwidth}
  \begin{center}
    \begin{table}[h]
      \begin{tabular}{ | c | c | }
        \hline
        \textbf{Function Call} & \textbf{Result} \\ 
        \hline
        fib(2) & 1 \\
        \hline
        fib(3) & 2 \\
        \hline
        fib(4) & 3 \\
        \hline
      \end{tabular}
    \end{table}
    \vspace*{0.7cm}

    \textbf{Cached Results}
  \end{center}
\end{minipage}
\end{slide}

% Generic Programming

\begin{slide}{Generic}
\centering
\large \textbf{\textcolor{red}{Generic} Incremental Computation for Regular Datatypes}

\vspace*{.5cm}
\textbf{=}
\vspace*{.5cm}

Generic refers to \textit{datatype-generic programming}, which is a form of abstraction that allows defining functions that can operate on a large class of datatypes. 

\end{slide}

% Generic Programming Example

\begin{slide}{Generic Example}
\begin{haskell}
data List a = Nil | Cons a (List a) -- Haskell Notation [] | x : [] 

length :: List a -> Int
length Nil        = 0
length (Cons _ t) = 1 + length t

data Tree a = Leaf | Node (Tree a) a (Tree a)

length :: Tree a -> Int
length Leaf _       = 1
length (Node l _ r) = 1 + length l + length r
\end{haskell}
\end{slide}

\begin{slide}{Generic Example}
\begin{chaskell}
gLength :: (Generic f) => f a -> Int
gLength = ...
\end{chaskell}

A \textit{single} \texttt{length} function can be written, that can operate on lists, trees, and many other datatypes

\vspace*{0.5cm}
\begin{haskell}
> gLength (Cons 1 (Cons 2 (Cons 3 Nil))) -- List Int
    3

> gLength (Node Leaf 1 Leaf) -- Tree Int
    3
\end{haskell}
\end{slide}

% Regular Datatypes

\begin{slide}{Regular Datatypes}
\centering
\large \textbf{Generic Incremental Computation for \textcolor{red}{Regular Datatypes}}

\vspace*{.5cm}
\textbf{=}
\vspace*{.5cm}

Regular datatypes are recursive datatypes, which can only recurse into themselves, such as lists, binary trees, etc. 

\end{slide}

% Regular Datatypes Example

\begin{slide}{Regular Datatypes Example}
\textbf{Regular Datatypes}

\begin{haskell}
data List a = Nil | Cons a (List a)

data Tree a = Leaf | Node a (Tree a) (Tree a)
\end{haskell}

\vspace*{0.5cm}
\textbf{\textcolor{red}{Not} Regular Datatypes}

\begin{haskell}
data Tree a = Empty | Node a (Forest a)

data Forest a = Nil | Cons (Tree a) (Forest a)
\end{haskell}
\end{slide}

\begin{slide}{Summary}
\begin{center}
  \large \textbf{Generic Incremental Computation for Regular Datatypes}
\end{center}

\vspace*{0.4cm}
\begin{itemize}
  \item Improve performance by only recomputing changed input
  \item Using generic programming to define functionality for a large class of datatypes
  \item The class of datatypes are regular datatypes
\end{itemize}
\end{slide}