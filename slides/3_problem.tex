\section{Goal}

\begin{slide}{What does this solve?}

Goal: implement an incremental algorithm which performs better than the non-incremental algorithm for large regular datatypes

\end{slide}

% Example

\begin{slide}{Example Problem with Memoization}
\begin{haskell}
data Tree a = Leaf a | Node (Tree a) a (Tree a)

sumTree :: Tree Int -> Int
sumTree (Leaf x)     = x
sumTree (Node l x r) = x + sumTree l + sumTree r

exampleTree = Node (Node (Leaf 1) 3 (Leaf 2)) 5 (Node (Leaf 1) 3 (Leaf 2))
\end{haskell}

\begin{center}
\textbf{Visual representation}

\begin{minipage}{.1\textwidth}
\texttt{sumTree(}
\end{minipage}
\begin{minipage}{.2\textwidth}
\begin{center}
\begin{minted}{text}
     5 
   /   \
  3     3
 / \   / \
1  2   1  2
\end{minted}
\end{center}
\end{minipage}
\begin{minipage}{.1\textwidth}
\texttt{) = 17}
\end{minipage}
\end{center}
\end{slide}

%  -------------------

\begin{slide}{Example Problem with Memoization}
  
{\Large \textbf{Memoized} version of the \texttt{sumTree}}

\begin{minipage}{.45\textwidth}
\begin{center}
\textbf{Example Tree}

\vspace*{0.4cm}
\begin{minted}[escapeinside=!!]{text}
     5 
   /   \
  3     !\textcolor{red}{3}!
 / \   !\textcolor{red}{/ \texttt{\symbol{92}}}!
1  2   !\textcolor{red}{1  2}!
\end{minted}
\end{center}
\end{minipage}
\hfill
\begin{minipage}{.45\textwidth}
\begin{center}
\begin{tabular}{ | m{0.45\textwidth} | >{\centering\arraybackslash} m{1cm} | }
\hline
\textbf{Tree} & \textbf{Result} \\ 
\hline
\vspace{0.3cm}
\begin{minipage}[t]{.2\textwidth}
\begin{minted}{text}
      5 
    /   \
   3     3
  / \   / \
 1  2   1  2
\end{minted}
\end{minipage}
\vspace*{0.3cm} & 17 \\
\hline
\vspace{0.3cm}
\begin{minipage}[t]{.2\textwidth}
\begin{minted}{text}
      3 
     / \ 
    1   2
\end{minted}
\end{minipage}
\vspace{0.5em}  & 6 \\
\hline
\end{tabular}
\vspace*{0.7cm}

\textbf{Cached Results}
\end{center}
\end{minipage}
\end{slide}