\section{Generic Programming}

\begin{slide}{Pattern Functors}

\textbf{Primitive Type Constructors}

\begin{haskell}
data U r         = U                 -- Empty constructor
data I r         = I r               -- Recursive position
data K a r       = K a               -- Constant
data (f :+: g) r = L (f r) | R (g r) -- Sums (Choice)
data (f :*: g) r = (f r) :*: (g r)   -- Products (Combine)
\end{haskell}

\textbf{Pattern functor for the} \texttt{Tree} \textbf{datatype}

\begin{haskell}
data Tree a = Leaf a | Node (Tree a) a (Tree a)

type instance PF (Tree a) = K a                -- Leaf
                         :+: (I :*: K a :*: I) -- Node
\end{haskell}
\end{slide}

% \begin{slide}{Fixed-point}

% Pattern functors only represent a single layer of recursion
  
% \begin{haskell}
% data Fix f = In (f (Fix f))

% type Tree a = Fix (PF (Tree a))
% \end{haskell}

% Isomorphic

% \begin{haskell}
% from :: Tree a -> Fix (PF (Tree a))
% from (Leaf x)     = In (K x)
% from (Node l x r) = In (I (from l) :*: K x :*: I (from r)) 

% to :: Fix (PF (Tree a)) -> Tree a
% to (In (K x))                 = Leaf x
% to (In (I l :*: K x :*: I r)) = Node (to l) x (to r)
% \end{haskell}
% \end{slide}

\section{Generic Implementation}

\begin{slide}{Hashing Primitive Type Constructors}
\begin{haskell}

class Hashable f where
  hash :: f (Merkle g) -> Digest
\end{haskell}

\begin{itemize}
  \item The \texttt{f} represents the primitive type constructors (i.e., \inlinehaskell{U, I, K, :+:, :*:}) 
  \item The \inlinehaskell{Merkle g} is the type of the recursive position (i.e., \inlinehaskell{I r}). 
  \item \inlinehaskell{Merkle g} contains the \inlinehaskell{Digest} of its internal structure.
  \item The \inlinehaskell{hash} function only converts a single layer of the pattern functor.
\end{itemize}
\end{slide}

\begin{slide}{Hashing Primitive Type Constructors}
\begin{haskell}
instance Hashable U where
  hash _ = hash "U"

instance (Show a) => Hashable (K a) where
  hash (K x) = digestConcat [hash "K", hash x]
\end{haskell}
\end{slide}

\begin{slide}{Hashing Primitive Type Constructors}
\begin{haskell}
instance (Hashable f, Hashable g) => Hashable (f :+: g) where
  hash (L x) = digestConcat [hash "L", hash x]
  hash (R x) = digestConcat [hash "R", hash x]

instance (Hashable f, Hashable g) => Hashable (f :*: g) where
  hash (x :*: y) = digestConcat [hash "P", hash x, hash y]
\end{haskell}
\end{slide}

\begin{slide}{Hashing Primitive Type Constructors}
\begin{haskell}
class Hashable f where
  hash :: f (Merkle g) -> Digest


instance Hashable I where
  hash (I x) = digestConcat [digest "I", getDigest x]
    where
      getDigest :: Fix (f :*: K Digest) -> Digest
      getDigest (In (_ :*: K h)) = h
\end{haskell}
\end{slide}

\begin{slide}{Generic Merkle Tree}
\begin{haskell}
merkleG :: Hashable f => f (Merkle g) -> (f :*: K Digest) (Merkle g)
merkleG f = f :*: K (hash f)

merkle :: (Regular a, Hashable (PF a), Functor (PF a))
       => a -> Merkle (PF a)
merkle = In . merkleG . fmap merkle . from
\end{haskell}
\end{slide}

\begin{slide}{Cata Merkle}

\texttt{cata} means \textit{catamorphism} which is a generalization of a \textit{fold}. A fold combines the data structure into a single value (e.g., \texttt{sumTree} is a fold).  

\vspace*{0.4cm}
\begin{haskell}
cataMerkleState :: (Functor f, Traversable f)
                => (f a -> a) -> Merkle f -> State (HashMap Digest a) a
cataMerkleState alg (In (x :*: K d)) 
  = do m <- get
       case lookup d m of
         Just a  -> return a
         Nothing -> do y <- mapM (cataMerkleState alg) x
                       let r = alg y
                       modify (insert d r) >> return r
\end{haskell}
\end{slide}

\begin{slide}{Cata Sum}
\begin{haskell}
cataSum :: Merkle (PF (Tree Int)) -> (Int, HashMap Digest Int)
cataSum = cataMerkle
  (\case
    L (K x)                 -> x
    R (I l :*: K x :*: I r) -> l + x + r
  )

> cataSum (merkle (Node (Leaf 1) 2 (Leaf 3)))
    (6, {"931090e5": 1, "7d1ef1c9": 3, "ba811ed5": 6})
\end{haskell}
\end{slide}

\begin{slide}{Pattern Synonyms}
\textit{Pattern synonyms} add an abstraction over patterns, which allows the user to move additional logic from \texttt{guards} and \texttt{case} expressions into patterns.

\vspace*{0.4cm}
\begin{haskell}
cataSum :: Merkle (PF (Tree Int)) -> (Int, HashMap Digest Int)
cataSum = cataMerkle
  (\case
    Leaf_ x     -> x
    Node_ l x r -> l + x + r
  )
\end{haskell}
\end{slide}