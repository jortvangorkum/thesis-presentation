\section{Solution}

\begin{slide}{Using Hash function}
A \textit{hash function} is a process of transforming input into an arbitrary fixed-size value (i.e., digest), where the same input always generates the same output

\begin{center}
\begin{minted}[escapeinside=!!]{text}
     5 
   /   \
  3     3       ->    1da16c7c48e429b4ad6e6c88d941d2bd
 / \   / \
1  2   1  2
\end{minted}
\end{center}
\end{slide}

\begin{slide}{Implementation Hash function}
\begin{haskell}
class Hashable a where
  hash :: a -> Digest

instance Hashable a => Hashable (Tree a) where
  hash (Leaf x)     = concatDigest [hash "Leaf", hash x]
  hash (Node l x r) = concatDigest [hash "Node", hash l, hash x, hash r]
\end{haskell}
\end{slide}

\begin{slide}{Storing the Digests}
A \textit{Merkle Tree} is a data structure which integrates the \textit{digests} within the data structure

\vspace*{0.4cm}
\begin{haskell}
data TreeH a = LeafH Digest a
             | NodeH Digest (TreeH a) a (TreeH a)


merkle :: Tree Int -> TreeH Int
merkle l@(Leaf x)     = LeafH (hash l) x
merkle b@(Node l x r) = NodeH  (hash b) l' x r'
  where
    l' = merkle l
    r' = merkle r
\end{haskell}
\end{slide}

\begin{slide}{Using Digests} 
\begin{haskell}
sumTreeInc m (LeafH d x) = case lookup d m of
  Just z  -> (z, m)
  Nothing -> (x, insert d x m)

...
\end{haskell}

\begin{haskell}
SumTreeInc 
\end{haskell}
\end{slide}

\begin{slide}{Updating the Digests}
The \textit{Zipper} is a technique for keeping track of how the data structure is being traversed through

\begin{haskell}
todo
\end{haskell}
\end{slide}