\section{Solution}

% Example

\begin{slide}{Example with Memoization}
\begin{haskell}
data Tree a = Leaf a | Node (Tree a) a (Tree a)

sumTree :: Tree Int -> Int
sumTree (Leaf x)     = x
sumTree (Node l x r) = x + sumTree l + sumTree r

exampleTree = Node (Node (Leaf 1) 3 (Leaf 2)) 5 (Node (Leaf 1) 3 (Leaf 2))
\end{haskell}

\begin{center}
\textbf{Visual representation}

\begin{minipage}{.1\textwidth}
\texttt{sumTree(}
\end{minipage}
\begin{minipage}{.2\textwidth}
\begin{center}
\begin{minted}{text}
     5 
   /   \
  3     3
 / \   / \
1  2   1  2
\end{minted}
\end{center}
\end{minipage}
\begin{minipage}{.1\textwidth}
\texttt{) = 17}
\end{minipage}
\end{center}
\end{slide}

%  -------------------

\begin{slide}{Example with Memoization}
  
{\Large \textbf{Memoized} version of the \texttt{sumTree}}

\begin{minipage}{.45\textwidth}
\begin{center}
\textbf{Example Tree}

\vspace*{0.4cm}
\begin{minted}[escapeinside=!!]{text}
     5 
   /   \
  3     !\textcolor{red}{3}!
 / \   !\textcolor{red}{/ \texttt{\symbol{92}}}!
1  2   !\textcolor{red}{1  2}!
\end{minted}
\end{center}
\end{minipage}
\hfill
\begin{minipage}{.45\textwidth}
\begin{center}
\begin{tabular}{ | m{0.45\textwidth} | >{\centering\arraybackslash} m{1cm} | }
\hline
\textbf{Tree} & \textbf{Result} \\ 
\hline
\vspace{0.3cm}
\begin{minipage}[t]{.2\textwidth}
\begin{minted}{text}
      5 
    /   \
   3     3
  / \   / \
 1  2   1  2
\end{minted}
\end{minipage}
\vspace*{0.3cm} & 17 \\
\hline
\vspace{0.3cm}
\begin{minipage}[t]{.2\textwidth}
\begin{minted}{text}
      3 
     / \ 
    1   2
\end{minted}
\end{minipage}
\vspace{0.5em}  & 6 \\
\hline
\end{tabular}
\vspace*{0.7cm}

\textbf{Cached Results}
\end{center}
\end{minipage}
\end{slide}

\begin{slide}{Using Hash function}
A \textit{hash function} is a process of transforming input into an arbitrary fixed-size value (i.e., digest), where the same input always generates the same output

\begin{center}
\begin{minted}[escapeinside=!!]{text}
     5 
   /   \
  3     3       ->    1da16c7c48e429b4ad6e6c88d941d2bd
 / \   / \    
1  2   1  2                       !\textcolor{red}{Digest}!  
\end{minted}
\end{center}

\vspace*{0.4cm}
The comparison for equality is now \textbf{constant} time instead of \textit{linear}.

\textbf{However}, the digest needs to be computed for every comparison, while the structure stays the same. So, the digest needs to be stored somewhere. 
\end{slide}

% \begin{slide}{Implementation Hash function}
% \begin{haskell}
% class Hashable a where
%   hash :: a -> Digest

% instance Hashable a => Hashable (Tree a) where
%   hash (Leaf x)     = concatDigest [hash "Leaf", hash x]
%   hash (Node l x r) = concatDigest [hash "Node", hash l, hash x, hash r]
% \end{haskell}
% \end{slide}

\begin{slide}{Storing the Digests}
A \textit{Merkle Tree} is a data structure which integrates the \textit{digests}, which represents the internal structure, within the data structure

\vspace*{0.4cm}
\begin{haskell}
data TreeH a = LeafH Digest a
             | NodeH Digest (TreeH a) a (TreeH a)


merkle :: Tree Int -> TreeH Int
merkle l@(Leaf x)     = LeafH (hash l) x
merkle b@(Node l x r) = NodeH (hash b) l' x r'
  where
    l' = merkle l
    r' = merkle r
\end{haskell}
\end{slide}

\begin{slide}{Efficiently updating the Input}

When a new tree is given to the function, the digests of the entire tree needs to be rehashed. This is \textbf{inefficient}, because, the affected digests are the changed parts of the tree and their parent nodes.

A more \textbf{efficient} method is, by passing a set of instructions on how to update the current input into the newly changed input. Using the set of instructions we can: 
\begin{itemize}
  \item traverse through the data structure
  \item modify the value of the current position with a given function
  \item and to return to a complete tree, we traverse back to the rootnode, while updating the digests
\end{itemize}

To implement the previously described functionality, we use a \textit{Zipper}. 
\end{slide}

\begin{slide}{Efficiently updating the Input}
\begin{center}
\begin{minted}[escapeinside=!!]{text}
     !\textcolor{red}{5}!                                5    
   /   \                               \   
  3     3      - !\textcolor{blue}{Left}! ->      !\textcolor{red}{3}!         3      - !\textcolor{blue}{Modify}! ->
 / \   / \                   / \       / \ 
1  2   1  2                 1   2     1   2
    
    
         5                       !\textcolor{red}{5}!     
          \                    /   \   
  !\textcolor{red}{7}!        3     - !\textcolor{blue}{Up}! ->      7     3   
 / \      / \                / \   / \ 
4   3    1   2              4  3   1  2
\end{minted}
\end{center}
\end{slide}

% \begin{slide}{Zipper - Left}


% \vspace*{0.4cm}
% \begin{haskell}
% left :: Loc a -> Loc a
% left (Node l x r, cs) = (l, (L r x):cs)
% \end{haskell}
% \end{slide}

% \begin{slide}{Zipper - Right}
% \begin{center}
% \begin{minted}[escapeinside=!!]{text}
%      !\textcolor{red}{5}!                    5     
%    /   \                 /           
%   3     3      ->       3          !\textcolor{red}{3}!  
%  / \   / \             / \        / \ 
% 1  2   1  2           1   2      1   2
% \end{minted}

% \vspace*{0.4cm}
% \begin{haskell}
% right :: Loc a -> Loc a
% right (Node l x r, cs) = (r, (R l x):cs)
% \end{haskell}
% \end{center}
% \end{slide}

% \begin{slide}{Zipper - Up}
% \begin{center}
% \begin{minted}[escapeinside=!!]{text}
%     5                        !\textcolor{red}{5}!        
%    /                       /   \           
%   3          !\textcolor{red}{3}!     ->     3     3      
%  / \        / \          / \   / \     
% 1   2      1   2        1  2   1  2   
% \end{minted}

% \vspace*{0.4cm}
% \begin{haskell}
% up :: Loc a -> Loc a
% up (t, (L r x):cs) = (Node l x r, cs)
% up (t, (R l x):cs) = (Node l x r, cs)
% \end{haskell}
% \end{center}
% \end{slide}